\documentclass{article}
\usepackage{graphicx}
\graphicspath{ {./images/} }
\usepackage{fontspec}
\usepackage{polyglossia}
\usepackage{amsmath}
\usepackage{float}
\setmainlanguage{hebrew}
\setmainfont{Times New Roman}
% \newfontfamily{\hebrewfont}{New Peninim MT}
\renewcommand*{\thesection}{}
\renewcommand*{\thesubsection}{\alph{subsection}.}
\renewcommand*{\thesubsubsection}{\arabic{subsubsection}.}
\title{תרגיל בית 1}
\date{}
\begin{document}
\maketitle
\newpage
\section{שאלה 1}
\subsection*{2}
נגדיר את מרחב החיפוש 
$(S, O, I, G)$.
\begin{itemize}
\item[$S$:]
קבוצת המצבים במרחב מייצגים את מיקום הסוכן ואילו כדורי דרקון נאספו. לכן כל מצב מוגדר ע"י מספר התא בו נמצא הסוכן והאם כדור 1 או 2 נאספו.
$$S = ([63]\cup\{0\}) \times \{0,1\} \times \{0, 1\}$$
\item[$O$:]
$$O = \{Down, Up, Left, Right\}$$
\item[$I$:]
$$I = \{(0, False, False)\}$$
\item[$G$:]
$$G = \{(63, True, True)\}$$
\end{itemize}
גודל מרחב המצבים הוא
$$64 \times 2 \times 2 = 256$$

\subsection*{3}
ניתן להפעיל Up בכל מצב חוץ מבחור, לכן הפונקציה Domain על אופרטור Up מוגדרת:
$$Domain(Up) = \{s \in S| board(s[0]) \neq H\}$$
כאשר board הוא לוח המשחק שמיוצג ע"י מחרוזת באורך 64.

\subsection*{4}
מהמצב ההתחלתי ניתן או לנסות לנוע למעלה או שמאלה ואז להישאר במקום, או לנוע ימינה למצב 1, או לנוע למטה למצב 8. לכן
$$Succ(0) = \{0, 1, 8\}$$

\subsection*{5}
במרחב החיפוש שלנו אכן קיימים מעגלים. לדוגמה ניתן ממצב 0 לנוע ימינה למצב 1, ואז לנוע שמאלה למצב 0 ולסגור מעגל.
\subsection*{6}
מקדם הסיעוף בבעיה הוא 4 כיוון שממצב מסוים ניתן לנוע לכל היותר לארבעה מצבים שונים (ויש מצב בו ניתן לנוע לארבעה מצבים שונים, למשל מצב 9).
\subsection*{7}
במקרה הגרוע ביותר, סוכן כללי לא יגיע למצב הסופי. למשל יכול להיות סוכן שנתקע במעגל לנצח.
\subsection*{8}
במקרה הטוב ביותר, ידרשו לסוכן 16 פעולות. מסלול שמגיע למצב הסופי צריך לאסוף את שני כדורי הדרקון ואז להגיע ל-G. לכן מסלול מינימלי הוא מסלול מינימלי מ-S לכדור דרקון כלשהו ואז מסלול מינימלי מכדור הדרקון לכדור הדרקון האחר, ולבסוף מסלול מינימלי מכדור הדרקון ל-G. ראינו בהרצאה ש-BFS מחזיר מסלול קצר ביותר. לכן נריץ שני מסלולים עם BFS והקצר מבניהם יהיה המסלול הקצר ביותר אל המצב הסופי:
\begin{enumerate}
\item $S \rightarrow dragonBall1 \rightarrow dragonBall2 \rightarrow G$
\item $S \rightarrow dragonBall2 \rightarrow dragonBall1 \rightarrow G$
\end{enumerate}
לאחר ההרצה קיבלנו שהמסלול הקצר הוא באורך של 16 פעולות.

\subsection*{9}
\end{document}